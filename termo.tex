\documentclass[a4paper, 12pt, oneside]{article}
\usepackage[T1]{fontenc}                     
\usepackage[utf8]{inputenc}                  
\usepackage{multirow}
\usepackage{fancyhdr}
\usepackage[magyar]{babel}                   
\usepackage{graphicx}                        
\usepackage{mathtools}
\usepackage{textgreek}
\usepackage{natbib}
\usepackage{gensymb}
\usepackage{mathtools}
\usepackage{bm}
\usepackage{esvect}

\title{Fajhő mérése}
\author{Mig András}
\date{2019.09.22}
\pagestyle{fancy}
\rhead{ 2019.09.16 Hétfő}
\lhead{Mig András}

\begin{document}

\begin{titlepage}
\begin{center}
    \LARGE{\textbf{Fajhő mérése}}\\
    \vspace{0.5cm}
    Mig András\\
    2019 Szeptember 16.
\end{center}
\begin{table}[!b]
\begin{tabular}{ll}
    A méréseket végezte:  & Mig András  \\
    A mérések dátuma: & 2019.09.16. 
\end{tabular}
\end{table}
\end{titlepage}


\newpage\null\thispagestyle{empty}\newpage
\part*{A mérés}
\subsection*{Cél}
    A mérés célja, hogy egy ismeretlen anyagú test fajhőjét meghatározzuk különböző módszerekkel. A mérés során a beejtés illetve az együtt melegítés módszerét is alkalmazzuk.
\subsection*{Eszközök}
\begin{enumerate}
    \item Digitális voltmérő
    \item Elektromos kaloriméter
    \item Hőkulcs
    \item Minták
    \item Kaloriméter tápegység
    \item Termosztát
\end{enumerate}
\section* {A mérés menete}


\subsection*{Fűtés}
    A kaloriméter melegítését egy feszültség alá helyezett fűtőtesttel végezzük. Ideális esetben, ha a környezet nem vesz fel hőt:
\begin{equation}
    Q = \frac{U^2}{R}t
\end{equation}
\subsection*{Korrigált hőmérséklet}
    Egy valódi kaloriméter nem képes hőveszteség nélkül tárolni a meleget, folyamatosan fennáll közte és a környezet között a hőcsere. A számítások azonban olyan adatot igényelnek, ami független ettől a kihüléstől. Ezért a mérés kezdetétől az adott időpillanatig összeintegráljuk a leadott hőt, azaz:
\begin{equation}
        f(t) = \int_0^t Q_\textrm{le}(t)
\end{equation}
    Ezt hozzáadjuk a mindenkori hőmérséklethez, és így megkaphatjuk a korrigált hőmérsékletet. A mérés során ezt a számítást szoftveresen végezzük el.
\subsection*{Hőátadás}
    Mivel a kaloriméter nem ideális, közötte és a minta között, valamint a környezete között folyamatos hőcsere alakul ki, ezt az ún. Newton-féle lehűlési törvénnyel írhatjuk le:
\begin{equation}
    \frac{dQ_{\textrm{h}}}{dT} = - h(T_{1} - T_{2}),
\end{equation}
    ahol a $h$-t hőátadási együtthatónak nevezzük.
\subsection*{Vízértékének meghatározása}
    A felvett hő miatt megváltozik a kaloriméter hőmérséklete. A megváltozás mértéke alapján meghatározhatjuk a vízértéket:
\begin{equation}
    v = \frac{Q}{\Delta T}
\end{equation}
\subsection*{Beejtéses módszer}
    A mintát előmelegítve bele helyezzük a kaloriméterbe. A kaolriméter által felvett hő megegyezik a mintaáltal leadott hővel, tehát:
\begin{equation}
    v(T_e-T_\textrm{körny}) = w(T_{m0}-T_e),
\end{equation}
    ahol $T_e$ a kiegyenlítődés utáni közös hőmérséklet, $T_{m0}$ a minta kezdeti hőmérséklete, $w$ pedig a minta hőkapacitása. Felhasználva a $w = c \cdot m$ és az előző egyenletet átrendezve a fajlagoshőkapacitásra:
\begin{equation}
        c = \frac{v}{m}\frac{T^*-T_\textrm{körny}}{T_{m0} - T_m^*}
\end{equation}
A hőmérővel csak a kaloriméter hőmérsékletét tudjuk mérni, a mintáét nem. Mivel a hőcsere nem ideális, ez a két érték nem egyezik meg. A minta kottigált hőmérsékletét a következő módon kaphatjuk meg:
\begin{equation}
    T_{m}^* =T_{\textrm{körny}}+\frac{\epsilon'}{\epsilon'-\epsilon_0}(T^*-T_\textrm{körny}),
\end{equation}
ahol $\epsilon'$ a főszakaszra illesztett exponenciális függvény kitevőjében szereplő állandó, $\epsilon_0$ pedig az üres kaloriméter utószakaszára illeszthető függvényé.
\subsection*{Együtt melegítéses módszer}
    A mérés menete megegyezik a vízérték meghatározásának menetével, azonban a mintát is belehelyezzük a kaloriméterbe. Ebben az esetben:
\begin{equation}
    (v + w) = \frac{Q}{\Delta T}
\end{equation}
    Figyelemmel arra, hogy nem tökéletes a hővezetés, felhasználva a $w = c \cdot m$ és az előző egyenletet átrendezve a fajlagoshőkapacitásra:
\begin{equation}
    c = \frac{1}{m} \frac{Q-v (T^*-T_\textrm{körny})}{T_{m}^*-T_\textrm{körny}}
\end{equation}
    

\subsection*{Vízérték mérése}
    Indítsuk el a hőmérő szoftvert, majd 2-3 perc elteltével kezdjük el fűteni a kalorimétert! Miután körül-belül 2,5 $^{\circ}$C-kal felmelegedett, kapcsoljuk ki és figyeljük meg a kaloriméter hülését! 15 perc elteltével állítsuk le a mérést és a hőkulcs segítségével hűtsük le a kiindulási hőmérsékletre! A mért adatok grafikonja az elsőszámú mellékleten látható.
\subsection*{Fajhő mérése beejtés módszerével}
    Helyezzük a termosztátba az egyik mintát! 30 perc elteltével indítsuk el a hőmérséklet mérését, majd rakjuk a kaloriméter felé a termosztátot  és abból ejtsük bele a mintát a kaloriméterbe! 15 perc elteltével állítsuk le a mérést! Vegyük ki a mintát és a hőkulcs segítségével hűtsük le a kalorimétért! A mért adatok grafikonja a második mellékleten látható.
\subsection*{Fajhő mérése az együtt fűtés módszerével}
    Helyezzük a lehűlt mintát a kaloriméterbe! Indítsuk el a hőmérő szoftvert, majd 2-3 perc elteltével kezdjük el fűteni a kalorimétert! Miután körül-belül 2,5 $^{\circ}$C-kal felmelegedett, kapcsoljuk ki és figyeljük meg a kaloriméter hülését! 15 perc elteltével állítsuk le a mérést! A mért adatok grafikonja az elsőszámú mellékleten látható.
\section*{Mért adatok kiértékelése}
\subsection*{Kiindulási adatok feljegyzése}
    A mérés megkezdése előtt néhány fontos adatot feljegyzünk. 
\begin{itemize}
  \item A rendszer ellenállása $ R_f = (7,07  \pm 0,01)  \textrm{ } \Omega $
  \item A fűtőfeszültség $ U_{\textrm{fűt}} = (1844 \pm 0,5)\textrm{ } \textrm{mV} $
  \item A test kezdeti hőmérséklete a beejtés során $T_{m0} = (34 \pm 0,5) \textrm{ } ^\circ \textrm{C}$
  \item A minta tömege $m = (4,781 \pm 0,0005) \textrm{ g}$
  \item Két mérés között eltelt idő $m = (1,23 \pm 0,005) \textrm{ s}$
\end{itemize}
    Ezek a hibák leolvasási hibák.\bigskip \newline 
    A méréseket három részre lehet osztani. Az előszakasz, ami a fűtés előtti hőmérsékletet hivatott mutatni, a főszakasz, ami alatt a fűtjük a kalorimétert és az utószakasz, amikor már hűl a rendszer. Az adatok kiértékelése egy erre a feladatra fejlesztett program segítségével történik. A mért értékek segítségével megállapíthatjuk, meddig tartott az előszakasz és mikor kezdődik az utó.
    A környezet hőmérsékletét $T_\textrm{körny}$-nyel, a fűtés kezdettét (a 2. mérésnél a beejtés pillanatát) $t_1$-gyel, annak végét $t_2$-vel, az utószakaszt leíró exponenciális egyenlet kitevőjében szereplő állandót $\epsilon$-nal, a főszakaszéban szereplőt $\epsilon'$-nal, a korrigált hőmérsékletet pedig $T^*$-ral jelölöm.   
    \begin{table}[!h]
        \centering
        \begin{tabular}{|c|c|c|c|c|}
        \hline
                                   &1. mérés & 2. mérés & 3. mérés &Mértékegység\\
        \hline
             $T_\textrm{körny}$& 22,58   & 21,88    & 19,62    &${ ^\circ}$C\\
        \hline
             $t_1$             & 2,32   & 2,784    & 2,066    & perc\\
        \hline
             $t_2$             & 4,82   & -        & 5,125    & perc\\
        \hline
             $T^*$             & 25,69   & 23,91    & 22,88    & ${ ^\circ}$C\\
        \hline
             $\epsilon$        & 0,07592  & 0,08363  & 0,07576  & $\frac{1}{perc}$\\
        \hline
        $\epsilon '$            & -      & 2,907    & -        & $\frac{1}{perc}$\\
        \hline
        \end{tabular}
        \caption{Az adatok négy értékes jegyre vannak kerekítve, hibáik ettől fogva kerekítési hibák.}
    \end{table}
\subsection*{A vízérték meghatározása}
    Egy vezetőn folyó áram által termelt hőenergia az (1) egyenlet szerint:
    $$ Q = \frac{U^2}{R}t$$
    Ebben az esetben $t = t_2 - t_1 = (150 \pm 1,74 \textrm{ s}$. \\ Így az összes felvett hőenergia: $$Q = \frac{(1,844 \textrm{ V})^2}{7,07 \textrm{ }\Omega } \cdot 150 \textrm{ s} = (72,14 \pm 5,348) \textrm{ J}$$
    A kaloriméter vízértéke kiszámolható, a (4) összefüggés alapján. Fontos, hogy mivel nem ideális a kaloriméter, figyelembe kell vennünk a hőveszteséget és így a korrigált hőmérséklettel kell számolni, azaz $\Delta T = T^*-T_\textrm{körny}$. $$ v =  (23,2 \pm 88.7) \frac{\textrm{J}}{\textrm{K}}$$

\subsection*{Fajhő meghatározása ejtéses módszerrel}
    A minta korrigált hőmérsékletét a (7) összefüggés alapján kapjuk meg:  
        $$T_m^*=T_{\textrm{körny}}+\frac{\epsilon'}{\epsilon'-\epsilon_0}(T^*-T_\textrm{körny}) = (23,9 \pm 0,13) \textrm{ } ^\circ \textrm{C}$$

Ezekből a fajhőt a (6) egyenlettel kaphatjuk meg a fajhőt:
\begin{equation*}
    c =  (0,89 \pm 3,822) \textrm{ }\frac{\textrm{J}}{\textrm{g $^\circ$C}}
\end{equation*}

\subsection*{Fajhő meghatározása együtt melegítéssel}
    Ebben az esetben $t = t_2 - t_1 = 183.5 \textrm{ s}$. \\ Így az összes felvett hőenergia: $$Q = \frac{(1,846 \textrm{ V})^2}{7,07 \textrm{ }\Omega } \cdot 183.5 \textrm{ s} = (88,45 \pm 39,39)\textrm{ J}$$
    A minta korrigált hőmérsékletét a korábbi összefüggés szerint számolhatjuk. Az $\epsilon '$ értékét ezzel a méréssel nem tudjuk megállapítani, ezért a korábbi mérésre támaszkodunk. 
    $$T_m^*=T_{\textrm{körny}}+\frac{\epsilon'}{\epsilon'-\epsilon_0}(T^*-T_\textrm{körny}) = (22.97 \pm 0,12) \textrm{ }^\circ \textrm{C}$$
    A (9) egyenletből:
$$
    c= (0,8 \pm 0,843) \textrm{ }\frac{\textrm{J}}{\textrm{g $^\circ$C}}
$$

\section*{Diszkusszió}
    A feladatban egy ismeretlen anyagú minta fajhőjét kellett meghatározni két különböző mérési módszerrel. A számolt értékekben a nagyon nagy abszoltút hibát a négy értékes jegyre való kerekítés okozhatta, hiszen ez az egy-két ezrelékes hiba igen erősen növekszik minden szorzás illetve osztás elvégzése esetén.
    
    \noindent Láthatjuk, hogy a két megállapított fajhő a valóságban jóval a számolt hibán belül található, azonban így is jelentső a különbség. Ennek oka elsősorba leolvasási, illetve kerekítési hiba. A vizsgált minta, sűrűsége és színe alapján alumínium vagy valamilyen alumínium ötvözet lehetett. Ezek fajhője $0,9\textrm{ }\frac{\textrm{J}}{\textrm{g $^\circ$C}}$ körül van, amihez a beejtéses módszerrel meghatározott fajhő elfogadhatóan közel esik.

\end{document}
